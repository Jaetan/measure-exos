\documentclass[11pt,a4paper,twoside]{article}
\usepackage{mathtools}
\usepackage{amsfonts}
\usepackage{amssymb}
\usepackage{amsthm}
\usepackage{mathrsfs}
\usepackage[shortlabels]{enumitem}

\theoremstyle{definition}
\newcounter{excounter}
\setcounter{excounter}{6}
\newtheorem{exercise}[excounter]{Exercise}

\begin{document}

\begin{exercise}

  Let $\mathscr{S}$ be a collection of subsets of the set $X$. Show that for each $A$ in $\sigma ( \mathscr{S} )$,
  there is a countable subfamily $\mathscr{C}_0$ of $\mathscr{S}$ such that $A \in \sigma ( \mathscr{C}_0 )$. (Hint:
  let $\mathscr{A}$ be the union of the $\sigma$-algebras $\sigma ( \mathscr{C} )$, where $\mathscr{C}$ ranges over
  the countable subfamilies of $\mathscr{S}$, and show that $\mathscr{A}$ is a $\sigma$-algebra that satisfies
  $\mathscr{S} \subseteq \mathscr{A} \subseteq \sigma ( \mathscr{S} )$ and hence is equal to $\sigma ( \mathscr{S} )$.)

\end{exercise}

\begin{proof}

  Using the notations defined above, for all $\sigma ( \mathscr{C} )$, we have $\varnothing \in \sigma ( \mathscr{C} )$,
  so $\varnothing \in \mathscr{A}$. Let $A \in \mathscr{A}$, there exists a countable subset $\mathscr{C}$ of $\mathscr{S}$
  such that $A \in \sigma ( \mathscr{C} )$. Since $\sigma ( \mathscr{C} )$ is stable by complementation,
  $A^c \in \sigma ( \mathscr{C} )$ and therefore $A^c \in \mathscr{A}$ (with $A^c = X - A$).

  Let now $\{ A_n \}_{n \in \mathbb{N}}$ be a family of subsets of $\mathscr{A}$. For all $n$, there exists a countable
  family $\mathscr{C}_n$ of subsets of $\mathscr{S}$, such that $A_n \in \sigma ( \mathscr{C}_n )$. The set
  $A = \cup_{n \in \mathbb{N}} \,A_n$ is an element of $\mathscr{T} = \cup_{n \in \mathbb{N}} \,\sigma ( \mathscr{C}_n )$,
  and for all $n \in \mathbb{N}$, $\mathscr{C}_n \subseteq \cup_{n \in \mathbb{N}} \,\mathscr{C}_n = \mathscr{C}$,
  so $\sigma ( \mathscr{C}_n ) \subseteq \sigma ( \mathscr{C} )$ and finally $\mathscr{T} \subset \sigma ( \mathscr{C} )$.
  Since $\mathscr{C}_n$ is countable for all $n$, we deduce that $\mathscr{C}$ is countable, and therefore that $A \in \mathscr{A}$.

  From the above, we conclude that $\mathscr{A}$ is a $\sigma$-algebra on $X$.

  ~\\
  For all $A \in \mathscr{A}$, we have $A \in \sigma ( \mathscr{C} )$ for some $\mathscr{C} \subseteq \mathscr{S}$.
  From this we deduce that $\sigma ( \mathscr{C} ) \subseteq \sigma ( \mathscr{S} )$, so that $\mathscr{A} \subseteq \sigma ( \mathscr{S} )$.
  For all $A \in \mathscr{S}$, the $\sigma$-algebra $\sigma ( A )$ is generated by the countable family $\{ A \}$,
  so $\sigma ( A ) \subseteq \mathscr{A}$. From this we deduce that $\mathscr{S} \subseteq \mathscr{A}$.

  From $\mathscr{S} \subseteq \mathscr{A} \subseteq \sigma ( \mathscr{S} )$, we deduce that $\mathscr{A}$ is included
  in the smallest $\sigma$-algebra that contains $\mathscr{S}$, and is therefore equal to $\sigma ( \mathscr{S} )$.
  
\end{proof}

\end{document}
