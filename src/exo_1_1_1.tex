\documentclass[11pt,a4paper,twoside]{article}
\usepackage{mathtools}
\usepackage{amsfonts}
\usepackage{amssymb}
\usepackage{amsthm}
\usepackage{mathrsfs}
\usepackage[shortlabels]{enumitem}

\theoremstyle{definition}
\newcounter{excounter}
\setcounter{excounter}{0}
\newtheorem{exercise}[excounter]{Exercise}

\begin{document}

\begin{exercise}

  Find the $\sigma$-algebra on $\mathbb{R}$ that is generated by all one-point subsets of $\mathbb{R}$.

\end{exercise}

\begin{proof}

  Let $\mathscr{C} = \big\{ \{ x \}, x \in \mathbb{R} \big\}$, and let $\mathscr{A} = \{ A \subset \mathbb{R}, A \text{ of } A^c \text{ is countable} \}$.
  Then $\mathscr{A}$ is a $\sigma$-algebra on $\mathbb{R}$: $\varnothing \in \mathscr{A}$ since it is finite, and by definition, $\mathscr{A}$ is
  stable by complementation.
  
  Let $A, B \in \mathscr{A}$. From $( A \cup B )^c = A^c \cap B^c$, we deduce that if $A^c$ or $B^c$ is countable, then $A \cup B \in \mathscr{A}$.
  Otherwise both $A$ and $B$ are countable, and then $A \cup B$ is also countable. Therefore for all $A, B \in \mathscr{A}$, $A \cup B \in \mathscr{A}$.

  Let $A$ be a countable set of elements of $\mathscr{A}$. Let $B = \{ a \in A, a \text{ is countable} \}$
  and $C = \{ a \in A, A^c \text{ is countable} \}$. As subsets of the countable set $A$, both $B$ and $C$ are countable, and we have $A = B \cup C$.
  The set $\cup_{x \in B} \,x$ is a countable union of countable sets, and therefore countable; and the set $\cup_{x \in C} \,x$ is such that
  \begin{equation*}
    \left( \bigcup_{x \in C} x \right)^c = \bigcap_{x \in C} x^c \subset x^c \text{ for all } x \in C
  \end{equation*}
  Since for all $x \in C$, $x^c$ is countable, $\cap_{x \in C} \,x^c$ is a subset of a countable set, and therefore countable. From this we deduce that
  \begin{equation*}
    \bigcup_{x \in A} x = \left( \bigcup_{x \in B} x \right) \cup \left( \bigcup_{x \in C} x \right)
  \end{equation*}
  is an element of $\mathscr{A}$ as the union of two elements of $\mathscr{A}$.

  ~\\
  Every element of $\mathscr{C}$ is finite and therefore an element of $\mathscr{A}$; thus we have $\mathscr{C} \subset \mathscr{A}$.
  Since $\mathscr{A}$ is a $\sigma$-algebra containing $\mathscr{C}$, we deduce that $\sigma ( \mathscr{C} ) \subset \mathscr{A}$.

  Conversely, let $A$ be a subset of $\mathscr{A}$. We have $A = \cup_{x \in A} \,\{ x \}$, from which we deduce that if $A$ is countable,
  then $A \in \sigma ( \mathscr{C} )$ as the countable union of elements of $\mathscr{C}$. Otherwise, $A^c = \cup_{x \in A^c} \,\{ x \}$ is countable,
  and $A = \cap_{x \in A^c} \,\{ x \}^c$ is the countable intersection of elements of $\sigma ( \mathscr{C} )$, since $\{ x \} \in \mathscr{C}$ for all $x \in \mathbb{R}$
  and $\sigma ( \mathscr{C} )$ is stable by complementation. Therefore $A \in \sigma ( \mathscr{C} )$.

  From the above, we deduce that $\sigma ( \mathscr{C} ) = \mathscr{A}$.

\end{proof}

\end{document}
