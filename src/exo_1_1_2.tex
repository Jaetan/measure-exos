\documentclass[11pt,a4paper,twoside]{article}
\usepackage{mathtools}
\usepackage{amsfonts}
\usepackage{amssymb}
\usepackage{amsthm}
\usepackage{mathrsfs}
\usepackage[shortlabels]{enumitem}

\theoremstyle{definition}
\newcounter{excounter}
\setcounter{excounter}{1}
\newtheorem{exercise}[excounter]{Exercise}

\begin{document}

\begin{exercise}

  Show that $\mathscr{B} ( \mathbb{R} )$ is generated by the collection of intervals $ ( {-\infty}, b ]$
    for which the endpoint $b$ is a rational number.

\end{exercise}

\begin{proof}

  Let $\mathscr{A} = \{ ( {-\infty}, r ], r \in \mathbb{Q} \}$ and $\mathscr{S} = \{ ( {-\infty}, x ], x \in \mathbb{R} \}$.
  We have $\mathscr{A} \subset \mathscr{S}$, so that $\sigma ( \mathscr{A} ) \subset \sigma ( \mathscr{S} ) = \mathscr{B} ( \mathbb{R} )$.

  Conversely, let $x \in \mathbb{R}$. For all $n \in \mathbb{N}$, let $r_n = x + 1 / ( n + 1 )$ and $I_n = ( {-\infty}, r_n ]$.
  Then $( {-\infty}, x ] = \cap_{n \in \mathbb{N}} \,I_n$ is the countable intersection of elements of $\mathscr{A}$ and therefore
  an element of $\sigma ( \mathscr{A} )$. From this we deduce that $\sigma ( \mathscr{S} ) \subset \sigma ( \mathscr{A} )$.

  Therefore $\mathscr{B} ( \mathbb{R} ) = \sigma ( \mathscr{A} )$.

\end{proof}

\end{document}
