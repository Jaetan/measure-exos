\documentclass[11pt,a4paper,twoside]{article}
\usepackage{mathtools}
\usepackage{amsfonts}
\usepackage{amssymb}
\usepackage{amsthm}
\usepackage{mathrsfs}
\usepackage[shortlabels]{enumitem}

\theoremstyle{definition}
\newcounter{excounter}
\setcounter{excounter}{2}
\newtheorem{exercise}[excounter]{Exercise}

\begin{document}

\begin{exercise}

  Show that $\mathscr{B} ( \mathbb{R} )$ is generated by the collection of all compact subsets of $\mathbb{R}$.

\end{exercise}

\begin{proof}

  Let $\mathscr{K}$ be the set of all compact subsets of $\mathbb{R}$ and $\mathscr{F}$ the set of all closed sets of $\mathbb{R}$.
  Let $F \in \mathscr{F}$, and for all $n \in \mathbb{N}$, let $F_n = F \cap [ {-n}, n ]$. As the intersection of closed sets,
  each $F_n$ is closed; and since it is also bounded, we conclude that $F_n$ is compact.
  Moreover, $F = \cup_{n \in \mathbb{N}} \,F_n$ is the countable union of elements of $\mathscr{K}$, so that $F \in \sigma ( \mathscr{K} )$.

  Conversely, every compact subset of $\mathbb{R}$ is closed, so $\mathscr{K} \subset \mathscr{F}$, and
  $\sigma ( \mathscr{K} ) \subset \sigma ( \mathscr{F} )$. Since $\mathscr{B} ( \mathbb{R} ) = \sigma ( \mathscr{F} )$,
  we conclude that $\mathscr{B} ( \mathbb{R} ) = \sigma ( \mathscr{K} )$.

\end{proof}

\end{document}
