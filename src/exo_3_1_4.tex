\documentclass[11pt,a4paper,twoside]{article}
\usepackage{mathtools}
\usepackage{amsfonts}
\usepackage{amssymb}
\usepackage{amsthm}
\usepackage{mathrsfs}
\usepackage[shortlabels]{enumitem}

\theoremstyle{definition}
\newcounter{excounter}
\setcounter{excounter}{3}
\newtheorem{exercise}[excounter]{Exercise}

\begin{document}

\begin{exercise}
  Suppose that $(X, \mathscr{A}, \mu)$ is a measure space and that $f$ and $f_1, f_2, \ldots$ belong to
  $\mathscr{L}^1 (X, \mathscr{A}, \mu, \mathbb{R})$. Show that if $\{ f_n \}$ converges to $f$ in mean so fast that
  \[
  \sum_n \int | f_n - f | \,\mathrm{d}\mu < +\infty
  \]
  then $\{ f_n \}$ converges to $f$ almost everywhere.
\end{exercise}

\begin{proof}
  Let $\epsilon > 0$, $N = \{ x \in X \mid f_n (x)$ does not converge to $f (x) \}$, and $A_k = \{ x \in X \mid | f_k (x) - f (x) | > \epsilon \}$.
  We have $N \subset \cap_{n = 1}^{\infty} \cup_{k = n}^\infty A_k$, so that
  \begin{align*}
    0 \leq \mu (N) &\leq \mu \left( \bigcap_{n = 1}^\infty \bigcup_{k = n}^\infty A_k \right)\\
    &\leq \mu \left( \bigcup_{k = n}^\infty A_k \right) \qquad\text{for any value of $n$}\\
    &\leq \sum_{k = n}^\infty \mu \left( A_k \right)\\
    &\leq \frac{1}{\epsilon} \sum_{k = n}^\infty \int | f_k - f | \,\mathrm{d}\mu \qquad\text{ by 2.3.14}\\
  \end{align*}
  The last term being the rest of a converging sum, we can choose $n$  large enough so that $\sum_{k = n}^\infty \int | f_k - f | \,\mathrm{d}\mu < \epsilon^2$,
  which leads to $0 \leq \mu (N) \leq \epsilon$. From this we deduce that $\{ f_n \}$ converges to $f$ almost everywhere.
\end{proof}

\end{document}
