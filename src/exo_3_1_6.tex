\documentclass[11pt,a4paper,twoside]{article}
\usepackage{mathtools}
\usepackage{amsfonts}
\usepackage{amssymb}
\usepackage{amsthm}
\usepackage{mathrsfs}
\usepackage[shortlabels]{enumitem}

\theoremstyle{definition}
\newcounter{excounter}
\setcounter{excounter}{5}
\newtheorem{exercise}[excounter]{Exercise}

\begin{document}

\begin{exercise}

  Let $\mu$ be a finite measure on $(A, \mathscr{A})$ and $f$ and $f_1, f_2, \dotsc$ be real-valude $\mathscr{A}$-measurable functions on $X$.
  Show that $\{ f_n \}$ converges to $f$ in measure if and only if each subsequence of $\{ f_n \}$ has a subsequence that converges to $f$ almost everywhere.

\end{exercise}

\begin{proof}\hfill

  Suppose that $\{ f_n \}$ converges to $f$ in measure and let $\{ g_n \}$ be a subsequence of $\{ f_n \}$.
  Then $\{ g_n \}$ also converges to $f$ in measure; therefore there exists a subsequence $\{ h_n \}$ of $\{ g_n \}$ that converges to $f$ almost everywhere.
  Since $\{ h_n \}$ is also a subsequence of $\{ f_n \}$, we have the first part of the result.

  Conversely, suppose that each subsequence of $\{ f_n \}$ has a subsequence that converges to $f$ almost everywhere, and suppose that $\{ f_n \}$ does not
  converge to $f$ in measure. There exists $\epsilon > 0$ and $\delta > 0$ such that
  \begin{equation}\label{eq:notinmeasure}
    \forall N \in \mathbb{N},\quad \exists n \geq N,\quad \mu \big( \{ x \in X \mid | f_n (x) - f (x) | > \delta \} \big) > \epsilon
  \end{equation}
  Using this inegality with $N = 0, 1, \dotsc$ we define a subsequence $\{ g_n \}$ of $\{ f_n \}$ such that $\mu \big( \{ x \in X \mid | g_n (x) - f (x) | > \delta \} \big) > \epsilon$
  for all $n \in \mathbb{N}$. Since $\{ g_n \}$ is a subsequence of $\{ f_n \}$, we can extract from it a subsequence $\{ h_n \}$ that converges to $f$ almost everywhere.
  Since $\mu$ is finite, $\{ h_n \}$ converges to $f$ also in measure. However, $\{ h_n \}$ being a subsequence of $\{ f_n \}$ must verify \eqref{eq:notinmeasure},
  which is a contradiction. From this, we deduce that $\{ f_n \}$ must converge to $f$ also in measure.\qedhere

\end{proof}

\end{document}
