\documentclass[11pt,a4paper,twoside]{article}
\usepackage{mathtools}
\usepackage{amsfonts}
\usepackage{amssymb}
\usepackage{amsthm}
\usepackage{mathrsfs}
\usepackage[shortlabels]{enumitem}

\theoremstyle{definition}
\newcounter{excounter}
\setcounter{excounter}{3}
\newtheorem{exercise}[excounter]{Exercise}

\begin{document}

\begin{exercise}

  Show that if $\mathscr{A}$ is an algebra of sets, and if $\cup_n \,A_n$ belongs to $\mathscr{A}$
  whenever $\{ A_n \}$ is a sequence of disjoint sets in $\mathscr{A}$, then $\mathscr{A}$ is a $\sigma$-algebra.

\end{exercise}

\begin{proof}

  Since $\mathscr{A}$ is an algebra, it contains $\varnothing$ and is stable by complementation. To show that
  $\mathscr{A}$ is a $\sigma$-algebra, it is enough to show that it is stable by countable unions.
  Let $\{ A_n \}_{n \in \mathbb{N}}$ be a family of sets of $\mathscr{A}$. Define a family $\{ B_n \}_{n \in \mathbb{N}}$
  of sets of $\mathscr{A}$ by
  \begin{align*}
    B_0 &= A_0 \\
    B_{n + 1} &= A_{n + 1} - \cup_{k = 0}^n \,A_k
  \end{align*}
  Then $\{ B_n \}_{n \in \mathbb{N}}$ is a family of disjoint sets of $\mathscr{A}$, so $\cup_{n \in \mathbb{N}} \,B_n \in \mathscr{A}$.
  However, $\cup_{n \in \mathbb{N}} \,B_n = \cup_{n \in \mathbb{N}} \,A_n$, so $\mathscr{A}$ is stable by countable unions,
  and is therefore a $\sigma$-algebra.

\end{proof}

\end{document}
