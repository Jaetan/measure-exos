\documentclass[11pt,a4paper,twoside]{article}
\usepackage{mathtools}
\usepackage{amsfonts}
\usepackage{amssymb}
\usepackage{amsthm}
\usepackage{mathrsfs}
\usepackage{cleveref}
\usepackage[shortlabels]{enumitem}
\usepackage{parskip}

\theoremstyle{definition}
\newcounter{excounter}
\setcounter{excounter}{5}
\newtheorem{exercise}[excounter]{Exercise}

\begin{document}

\begin{exercise}

  Show that the hypothesis of $\sigma$-finiteness cannot be omitted from Proposition 3.4.5.
  (\emph{Hint:} Consider counting measure on $( \mathbb{R}, \mathscr{B} ( \mathbb{R} ) )$.)

\end{exercise}

\begin{proof}

  Let $\mu$ be the counting measure on $( \mathbb{R}, \mathscr{B} ( \mathbb{R} ) )$.
  The countable set $\mathscr{A} = \{ ( {-\infty}, r ], r \in \mathbb{Q} \}$ satisfies
  $\sigma ( \mathscr{A} ) = \mathscr{B} ( \mathbb{R} )$ (for example from exercise 1.1.2),
  so $\mathscr{B} ( \mathbb{R} )$ is countably generated by $\mathscr{A}$.
  Let $\{ X_n \}_{n \in \mathbb{N}}$ be a family of Borel subsets of $\mathbb{R}$ such that
  $\mu ( X_i )$ is finite for all $i$. Since $\mu$ is the counting measure, $X_i$ is
  also finite for all $i$. As a countable union of finite sets, $\cup_{n \in \mathbb{N}} \,X_n$
  is countable; since $\mathbb{R}$ is uncountable, $\mathbb{R} \neq \cup_{n \in \mathbb{N}} \,X_n$,
  and therefore $\mu$ is not $\sigma$-finite.

  Let $\varepsilon > 0$ and suppose that there exists a family of functions $\{ \varphi_n \}_{n \in \mathbb{N}}$
  in $\mathscr{L}^p ( \mathbb{R}, \mathscr{B} ( \mathbb{R} ), \mu )$ such that:
  \begin{equation*}
    \forall f \in \mathscr{L}^p ( \mathbb{R}, \mathscr{B} ( \mathbb{R} ), \mu ), \quad\exists n \in \mathbb{N}, \quad \| f - \varphi_n \|_p < \varepsilon
  \end{equation*}
  Let $\{ \delta_x \}_{x \in \mathbb{R}}$ be the family of functions of $\mathscr{L}^p ( \mathbb{R}, \mathscr{B} ( \mathbb{R} ), \mu )$ defined by:
  \begin{equation*}
    \forall y \in \mathbb{R}, \quad \delta_x ( y ) = \begin{cases}
      1 &\text{if } y = x \\
      0 &\text{otherwise}
    \end{cases}
  \end{equation*}
  Let $x, y \in \mathbb{R}$ such that $x \neq y$; we have $\| \delta_x - \delta_y \|_p = 2^{1 / p} \geq 1$. If $m, n \in \mathbb{N}$ are such that
  \begin{align*}
    \| \delta_x - \varphi_n \|_p &< \varepsilon \\
    \| \delta_y - \varphi_m \|_p &< \varepsilon
  \end{align*}
  then the triangular inequality gives us
  \begin{equation} \label{ineq:delta}
    \| \varphi_n - \varphi_m \|_p \geq \| \delta_x - \delta_y \|_p - \big( \| \delta_x - \varphi_n \|_p + \| \delta_y - \varphi_m \|_p \big) > 1 / 3
  \end{equation}
  as soon as $\varepsilon < 1 / 3$. For all $x \in \mathbb{R}$, let $D_x = \{ n \in \mathbb{N}, \| \delta_x - \varphi_n \|_p < \varepsilon \}$.
  Using the axiom of choice, we can define a function
  \begin{align*}
    \psi : \{ \delta_x \}_{x \in \mathbb{R}} &\to \mathbb{N} \\
    \delta_y &\mapsto n \quad\text{such that } n \in D_y
  \end{align*}
  From \eqref{ineq:delta}, we deduce that $\psi$ is injective, and therefore $\{ \delta_x \}_{x \in \mathbb{R}}$ must be countable, which is absurd.
  Therefore, the family $\{ \varphi_n \}_{n \in \mathbb{N}}$ does not exist and $L^p ( \mathbb{R}, \mathscr{B} ( \mathbb{R} ), \mu )$ is not separable.

\end{proof}

\end{document}
