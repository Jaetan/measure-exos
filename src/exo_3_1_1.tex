\documentclass[11pt,a4paper,twoside]{article}
\usepackage{mathtools}
\usepackage{amsfonts}
\usepackage{amssymb}
\usepackage{amsthm}
\usepackage{mathrsfs}
\usepackage[shortlabels]{enumitem}

\theoremstyle{definition}
\newcounter{excounter}
\setcounter{excounter}{0}
\newtheorem{exercise}[excounter]{Exercise}

\begin{document}

\begin{exercise}
  Let $(X, \mathscr{A}, \mu)$ be a measure space, and let $A$ and $A_1, A_2, \ldots$ belong to $\mathscr{A}$. Show that
  \begin{enumerate}[(a)]
  \item
    $\{ \chi_{A_n} \}$ converges to 0 in measure if and only if $\lim_n \mu (A_n) = 0$.

  \item
    $\{ \chi_{A_n} \}$ converges to 0 almost everywhere if and only if $\mu (\cap_{n = 1}^\infty \cup_{k = n}^\infty A_k) = 0$

  \item
    $\{ \chi_{A_n} \}$ converges to $\chi_A$ almost everywhere if and only if the three sets $A$, $\cap_{n = 1}^\infty \cup_{k = n}^\infty A_k$ and $\cup_{n = 1}^\infty \cap_{k = n}^\infty A_k$
    differ only by $\mu$-null sets (Hint: see exercise 2.1.1)

    For reference, exercise 2.1.1 state that for $(A_k)$ a sequence of subsets of $X$, and $B = \cup_{n = 1}^\infty \cap_{k = n}^\infty A_k$ and $C = \cap_{n = 1}^\infty \cup_{k = n}^\infty A_k$,
    we have $\lim\inf_k \chi_{A_k} = B$ and $\lim\sup_k \chi_{A_k} = C$.
  \end{enumerate}
\end{exercise}

\begin{proof}\hfill
  \begin{enumerate}[(a)]
  \item
    Let $\epsilon > 0$, we have $\{ x \in X \mid | \chi_{A_n} (x) | > \epsilon \} = \{ x \in X \mid \chi_{A_n} (x) = 1 \} = A_n$, from which we deduce
    $\mu \big( \{ x \in X \mid | \chi_{A_n} (x) | > \epsilon \} \big) = \mu (A_n)$. So $\{ \chi_{A_n} \}$ converges in measure to 0 is equivalent to $\mu (A_n)$ converges to 0.

  \item
    Let $\epsilon > 0$ and $n \in \mathbb{N}$. As before, we note that $A_n = \{ x \in X \mid | \chi_{A_n} (x) | > \epsilon \}$.
    Define $B_n = \cup_{k = n}^\infty A_k$ and $B = \cap_{n = 1}^\infty B_n$, and remark that $\{ x \in X \mid \{ \chi_{A_n} (x) \}$ does not converge to 0 $\}$ is a subset of $B$.
    Therefore if $\mu (\cap_{n = 1}^\infty \cup_{k = n}^\infty A_k) = 0$, then $\{ \chi_{A_n} \}$ converges to 0 almost everywhere.

    Conversely, suppose that $\{ \chi_{A_n} \}$ converges to 0 almost everywhere, and choose a $\mu$-null set $N \subset \mathscr{A}$
    such that $\{ x \in X \mid \{ \chi_{A_n} \}$ does not converge to 0 $\} \subset N$. Define $D_n = A_n \cap N^C$; the series $(\chi_{D_n})$ converges simply to 0 everywhere.
    Let $D = \cap_{n = 1}^\infty \cup_{k = n}^\infty D_k$ and $x \in D$. Then for all $n$, $\exists k \geq n,\quad \chi_{D_n} (x) = 1$,
    which contradicts the simple convergence of $(\chi_{D_n})$ to 0 on $N^C$. It follows that $D = \varnothing$, and thus $\mu (D) = 0$.

    Next, define $S = \cap_{n = 1}^\infty \cup_{k = n}^\infty A_k$. We have:
    \begin{align*}
    \mu (S) &= \mu (S \cap N) + \mu (S \cap N^C)\\
    \mu (S \cap N) &\leq \mu (N) = 0\\
    \mu (S \cap N^C) &= \mu \left( \cap_{n = 1}^\infty \cup_{k = n}^\infty \left( A_k \cap N^C \right) \right) = \mu (D) = 0\\
    \end{align*}
    which gives us $\mu (\cap_{n = 1}^\infty \cup_{k = n}^\infty A_k) = 0$

  \item
    

  \end{enumerate}
\end{proof}

\end{document}
