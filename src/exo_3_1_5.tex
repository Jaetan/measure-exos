\documentclass[11pt,a4paper,twoside]{article}
\usepackage{mathtools}
\usepackage{amsfonts}
\usepackage{amssymb}
\usepackage{amsthm}
\usepackage{mathrsfs}
\usepackage[shortlabels]{enumitem}

\theoremstyle{definition}
\newcounter{excounter}
\setcounter{excounter}{4}
\newtheorem{exercise}[excounter]{Exercise}

\begin{document}

\begin{exercise}
  Let $\mu$ be a measure on $(X, \mathscr{A})$, and et $f, f_1, f_2, \dotsc$ and $g, g_1, g_2, \dotsc$ be real-valued
  $\mathscr{A}$-measurable functions on $X$.
  \begin{enumerate}[(a)]
  \item Show that if $\mu$ is finite, if $\{ f_n \}$ converges to $f$ in measure, and if $\{ g_n \}$ converges to $g$ in measure,
    then $\{ f_n g_n \}$ converges to $fg$ in measure.
  \item Can the assumption that $\mu$ is finite be omitted in part (a)?
  \end{enumerate}
\end{exercise}

\begin{proof}\hfill

  \begin{enumerate}[(a)]

  \item First, note that $\forall x \in \mathbb{R}$,
  \begin{multline*}
    f_n (x) g_n (x) - f (x) g (x) = \big( f_n (x) - f (x) \big) \big( g_n (x) - g (x) \big) \\
    + f (x) \big( g_n (x) - g (x) \big) + g (x) \big( f_n (x) - f (x) \big)
  \end{multline*}

  Let $\alpha > 0$. For all positive reals $0 < \alpha_1 < \alpha_2 < \alpha_3$ such that $\alpha = \alpha_1 + \alpha_2 + \alpha_3$,
  we have $\alpha_3 \geq \alpha / 3$ (otherwise their sum would be less than $\alpha$). From this we deduce that:
  \begin{multline}\label{eq:inclusion}
      \big\{ x \in X \mid | f_n (x) g_n (x) - f (x) g (x) | > 3 \alpha \big\} \subset \\
      \big\{ x \in X \mid | f_n (x) - f (x) | | g_n (x) - g (x) | > \alpha \big\} \\
      \cup \big\{ x \in X \mid | f (x) | | g_n (x) - g (x) | > \alpha \big\} \\
      \cup \big\{ x \in X \mid | g (x) | | f_n (x) - f (x) | > \alpha \big\}
  \end{multline}

  Now let $n \in \mathbb{N}$. Define $F_n = \{ x \in X \mid | f (x) | > n \}$ and $G_n = \{ x \in X \mid | g (x) | > n \}$.
  Since $\mu$ is finite, $\mu ( F_n )$ is finite, and we also have
  $F_{n + 1} \subset F_n$. Since $f$ is real-valued, we deduce:
  \begin{equation*}
    \mu \left( \bigcap_{n = 0}^\infty F_n \right) = \lim_{n \to \infty} \mu (F_n) = 0
  \end{equation*}
  The same holds for $g$ and $G_n$.

  Let $\epsilon > 0$. Choose $N \in \mathbb{N}$ such that $\mu ( F_n ) < \epsilon$ and $\mu ( G_n ) < \epsilon$ for all $n \geq N$.

  Note that
  \begin{multline*}
    \big\{ x \in X \mid | f (x) | \leq N \big\} \cap \big\{ x \in X \mid | g_n (x) - g (x) | \leq \alpha / N \big\} \\
    \subset \big\{ x \in X \mid | f (x) | | g_n (x) - g (x) | \leq \alpha \big\}
  \end{multline*}
  Taking the complement in $X$, we get
  \begin{multline*}
    \big\{ x \in X \mid | f (x) | | g_n (x) - g (x) | > \alpha \big\} \\
    \subset F_n \cup \big\{ x \in X \mid | g_n (x) - g (x) | > \alpha / N \big\}
  \end{multline*}
  which implies
  \begin{multline*}
    \mu \left( \big\{ x \in X \mid | f (x) | | g_n (x) - g (x) | > \alpha \big\} \right) \\
    \leq \mu (F_n) + \mu \left( \big\{ x \in X \mid | g_n (x) - g (x) | > \alpha / N \big\} \right)
  \end{multline*}

  Similarly, we have
  \begin{multline*}
    \mu \left( \big\{ x \in X \mid | g (x) | | f_n (x) - f (x) | > \alpha \big\} \right) \\
    \leq \mu (G_n) + \mu \left( \big\{ x \in X \mid | f_n (x) - f (x) | > \alpha / N \big\} \right)
  \end{multline*}

  Finally, we also have
  \begin{multline*}
    \big\{ x \in X \mid | f_n (x) - f (x) | \leq \sqrt{\alpha} \big\} \cap \big\{ x \in X \mid | g_n (x) - g (x) | \leq \sqrt{\alpha} \big\} \\
    \subset \big\{ x \in X \mid | f_n (x) - f (x) | | g_n (x) - g (x) | \leq \alpha \big\}
  \end{multline*}
  which implies
  \begin{multline*}
    \mu \left( \big\{ x \in X \mid | f_n (x) - f (x) | | g_n (x) - g (x) | > \alpha \big\} \right) \\
    \leq \mu \left( \big\{ x \in X \mid | f_n (x) - f (x) | > \sqrt{\alpha} \big\} \right) \\
    + \mu \left( \big\{ x \in X \mid | g_n (x) - g (x) | > \sqrt{\alpha} \big\} \right)
  \end{multline*}

  Summarizing the results above, we get:
  \begin{equation*}
    \begin{split}
    \mu & \left( \big\{ x \in X \mid | f_n (x) g_n (x) - f (x) g (x) | > 3 \alpha \big\} \right) \\
    &\leq \mu \left( \big\{ x \in X \mid | f_n (x) - f (x) | > \sqrt{\alpha} \big\} \right) \\
    &\quad{}+ \mu \left( \big\{ x \in X \mid | g_n (x) - g (x) | > \sqrt{\alpha} \big\} \right) \\
    &\quad{}+ \mu (F_n) + \mu \left( \big\{ x \in X \mid | g_n (x) - g (x) | > \alpha / N \big\} \right) \\
    &\quad{}+ \mu (G_n) + \mu \left( \big\{ x \in X \mid | f_n (x) - f (x) | > \alpha / N \big\} \right)
    \end{split}
  \end{equation*}

  The convergence of $\{ f_n \}$ (resp. $\{ g_n \}$) to $f$ (resp. $g$) in measure allows us to choose $M \in \mathbb{N}$ greater than $N$ and such that
  the 4 measures above involving $| f_n - f |$ and $| g_n - g |$ are simultaneously less than $\epsilon$ when $n \geq M$. From this we deduce that $\{ f_n g_n \}$
  converges to $f g$ in measure.

  \item The hypothesis that $\mu$ is finite is necessary. Consider for example for $x > 0$, $f_n (x) = \exp (x)$ and $g_n (x) = \frac{1}{x} + \frac{1}{n}$ for $n \geq 1$.
    We have $| g_n (x) - \frac{1}{x} | = \frac{1}{n}$; then $\{ g_n \}$ converges uniformly to $x \mapsto \frac{1}{x}$, and therefore converges also in measure.
    \begin{align*}
      f_n (x) g_n (x) - f (x) g (x) &= \exp (x) \left( \frac{1}{x} + \frac{1}{n} \right) - \frac{\exp (x)}{x} \\
      &= \frac{1}{n} \exp (x)
    \end{align*}
    Let $\epsilon > 0$, $\{ x > 0 \mid \frac{1}{n} \exp (x) > \epsilon \} = {[\ln (n \epsilon), +\infty)}$, whose Lebesgue measure is always $+\infty$,
      so $\{ f_n g_n \}$ does not converge in measure to $f g$.\qedhere

  \end{enumerate}

\end{proof}

\end{document}
